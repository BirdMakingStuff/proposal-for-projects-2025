\documentclass[../proposal-for-projects-2025.tex]{subfiles}

\begin{document}

\subsection{Free and Open-Source Admin Discord Bot}

This is a Discord bot that aims to be a free/libre and open-source replacement for popular server management bots like Carl.bot and MEE6.  

The project aims to be modular, with the core including a plugin manager that allows various capabilities to be added or removed on-demand by users. Some of these capabilities include, but are not limited to: logging \& moderation, allowing users to select their own roles through reactions, integration with Google Sheets to track club members.

A team of four to six people is desired.

\subsubsection*{Skills Desired}

The tech stack is still up for debate, particularly between a Python-based library (e.g. \texttt{discord.py}, Pycord) and JavaScript/TypeScript (e.g. Discord.js).  The following skills are desired, but not required:

\begin{multicols}{2}
\begin{itemize}
	\item Python (\texttt{discord.py}, Pycord)
    \item Node.js and TypeScript (Discord.js)
    \item HTTP REST API
    \item Functional Programming
    \item WebSockets
    \item Git and GitHub
    \item Discord server management features
\end{itemize}
\end{multicols}

\subsubsection*{Time Commitment}

Each person is expected to contribute about 40 to 60 hours in total, or roughly 5 hours per week for a duration of 10 weeks.

There will be a weekly stand up that is expected to take between 30 minutes and 1 hour per session, depending on the amount of items to be covered.

\subsubsection*{Key Benefits}

By participating in this project, you can expect to gain the following benefits:

\begin{itemize}
    \item Improved skills and knowledge in software architecture, event-driven programming, software testing and networking
    \item Improved communication skills
    \item Free/libre and open-source server management bot that you can use for your own servers
\end{itemize}

\end{document}
